\documentclass[letterpaper,12pt,oneside]{report} 


\usepackage{amssymb} 
\usepackage{amsmath}
\usepackage{tocbibind}
\usepackage{graphicx}
\usepackage[latin1]{inputenc}
\usepackage[spanish]{babel}  

\parindent=0pt  
\renewcommand{\baselinestretch}{1}
\setlength{\topmargin}{-2cm} \setlength{\textheight}{26cm}
\setlength{\oddsidemargin}{-0.5cm}
\setlength{\evensidemargin}{0.0in} \setlength{\textwidth}{17cm}
\thispagestyle{empty}



\begin{document}


\textbf{ACTIVIDAD DIGITAL No. 2 DE MATEM�TICAS DISCRETAS II\\
Profesora: Diana Bueno\\
Entregado por: Laura Blanco - Juliana Campo Jim�nez \\
\textbf{Fecha de entrega:} Lunes 12 de abril de 2021}

\vspace{0.2cm}
\vspace{-0.3cm}
\rule{17cm}{0.02cm}
\vspace{0.1cm}

\textbf{1.}Sea $f_{n}=5f_{n-4}+3f_{n-5}$ con $n=5,6,7,...$, con condiciones iniciales $f_{0}=0, f_{1}=1, f_{2}=1, f_{3}=2$ y $f_{4}=3$. Demuestre que $f_{5n}$ es divisible por 5, para $n = 1, 2, 3,...$\\

\textbf{Demostraci�n.} Empleando el principio fuerte de inducci�n matem�tica sobre n, tenemos:\\

\textit{Paso base:}  n = 1



\begin{equation*}
\begin{split}
f_{5(1)}&=5f_{5(1)-4}+3f_{5(1)-5}\\
        f_{5}&=5f_{1}+3f_{0}\\
        f_{5}&=5(1)+3(0) \\
        f_{5}&=5 
\end{split} 
\end{equation*}

Es f�cil ver que $f_{5}=5$, es m�ltiplo de 5.\\

\textit{Paso inductivo:} Por ver que $f_{5n}=5f_{5n-4}+3f_{5n-5}$ con $n=5,6,7,...$ es m�ltiplo de 5.\\

Supongamos por hip�tesis de inducci�n que se cumple $f_{5n}=5f_{5n-4}+3f_{5n-5}$ con $n=2,3,...,k.$\\

Debemos demostrar que $f_{5(k+1)}$ es m�ltiplo de 5.\\

\begin{equation*}
\begin{split}
f_{5(k+1)}&=5f_{5(k+1)-4}+3f_{5(k+1)-5}\\
f_{5k+5}&= 5f_{5k+1}+3f_{5k}        
\end{split} 
\end{equation*}\\

Por hip�tesis de inducci�n $f_{5k}$ es m�ltiplo de 5, por lo tanto, $3f_{5k}$ es tambi�n m�ltiplo de 5. Adem�s, $5f_{5k+1}$ es m�ltiplo de 5, ya que cualquier n�mero multiplicado por 5 es m�ltiplo de 5.\\

Ahora bien, la suma de dos m�ltiplos de 5 es un m�ltiplo de 5, por lo tanto $f_{5k+5}$ es m�ltiplo de 5.


\end{document}

